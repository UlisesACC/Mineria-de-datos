
\documentclass[12pt]{article}
\usepackage[margin=1in]{geometry}
\usepackage{longtable}
\usepackage{array}
\usepackage{hyperref}
\usepackage{bookmark}
\usepackage{float}
\usepackage{graphicx}
\usepackage{amsmath}

\begin{document}

\begin{center}
{\LARGE INSTITUTO POLITÉCNICO NACIONAL}\\
{\Large ESCUELA SUPERIOR DE CÓMPUTO}\\
{\Large \textbf{PRACTICA DE CLASIFICACIÓN (Bagging, Boosting)}}\\
\end{center}

\vspace{1cm}

\noindent\textbf{Grupo:} \_\_\_\_\_\_\_\_\\
\textbf{No. Equipo:} \_\_\_\_\_\_\_\_\\
\textbf{Nombres:}
Ulises Abdiel Cabello Cardenas\\
Fernanda Madariaga Villanueva

\vspace{1cm}

\section*{Descripción del conjunto de datos}
Reconocimiento de Género por Voz. Identificar una voz como masculina o femenina.

\noindent\textbf{No. Licencia:} CC BY-NC-SA 4.0\\
\textbf{Disponible en:} \url{https://www.kaggle.com/datasets/primaryobjects/voicegender/data}

\section*{Diccionario de datos}
El conjunto de datos cuenta con 22 atributos, los cuales se describen en la siguiente tabla.

\begin{table}[H]
\centering
\begin{tabular}{lll}
\hline
\textbf{Variable} & \textbf{Tipo} & \textbf{Significado} \\
\hline
meanfreq & numérico & Frecuencia media (kHz) \\
sd & numérico & Desviación estándar de la frecuencia \\
median & numérico & Frecuencia mediana (kHz) \\
Q25 & numérico & Primer cuantil (kHz) \\
Q75 & numérico & Tercer cuantil (kHz) \\
IQR & numérico & Rango intercuartílico (kHz) \\
skew & numérico & Asimetría del espectro \\
kurt & numérico & Curtosis del espectro \\
sp.ent & numérico & Entropía espectral \\
sfm & numérico & Planitud espectral \\
mode & numérico & Frecuencia de la moda \\
centroid & numérico & Centroide de frecuencia \\
peakf & numérico & Frecuencia pico (mayor energía) \\
meanfun & numérico & Promedio de frecuencia fundamental \\
minfun & numérico & Frecuencia fundamental mínima \\
maxfun & numérico & Frecuencia fundamental máxima \\
meandom & numérico & Promedio de frecuencia dominante \\
mindom & numérico & Frecuencia dominante mínima \\
maxdom & numérico & Frecuencia dominante máxima \\
dfrange & numérico & Rango de frecuencia dominante \\
modindx & numérico & Índice de modulación \\
label & categórico & Género: masculino o femenino \\
\hline
\end{tabular}
\end{table}


\section*{Consideraciones encontradas en el conjunto de datos}
Describir las consideraciones que encuentre en el conjunto de datos.


\begin{figure}[H]
\centering
\includegraphics[width=0.8\textwidth]{consdata.jpg}
\caption{Consideraciones del conjunto de datos}
\end{figure}
\begin{figure}[H]
\centering
\includegraphics[width=0.8\textwidth]{consdata2.jpg}
\caption{Consideraciones del conjunto de datos}
\end{figure}

\begin{itemize}
    \item \textbf{Tipos de datos:}  
    Todos los atributos predictivos del conjunto de datos son de tipo \textit{numérico (float)}, mientras que la variable objetivo \texttt{label} es \textit{categórica} con dos clases: \texttt{male} y \texttt{female}. Esto facilita el uso de modelos basados en árboles, pues no requieren codificación adicional ni normalización estricta.

    \item \textbf{Valores faltantes:}  
    Ninguna variable contiene valores faltantes:  
    $
    \text{Missing Values} = 0
    $
    Por lo tanto, no se necesitó realizar imputación ni eliminación de registros.

    \item \textbf{Número de valores únicos:}  
    Varias variables presentan un número muy alto de valores únicos (hasta 3166), lo que indica un comportamiento continuo.  
    Sin embargo, se observaron variables con menor variabilidad, tales como:
    \begin{itemize}
        \item \texttt{maxfun}: 123 valores únicos
        \item \texttt{mindom}: 77 valores únicos
        \item \texttt{minfun}: 913 valores únicos
        \item \texttt{dfrange}: 1091 valores únicos
    \end{itemize}
    Estas diferencias sugieren que algunas características espectrales poseen menor dispersión que otras.

    \item \textbf{Presencia de outliers:}  
    Aunque no existen valores faltantes, se identifica la presencia de valores extremos en variables como:
    \texttt{skew}, \texttt{kurt}, \texttt{maxdom} y \texttt{dfrange}.  
    Debido a que los modelos de tipo árbol (Random Forest y Boosting) son robustos ante outliers, se decidió no realizar eliminación ni transformación de estos valores.

    \item \textbf{Balance de clases:}  
    El conjunto de datos está perfectamente balanceado con:
    $
    1584\ \text{voces masculinas} \quad \text{y} \quad 1584\ \text{voces femeninas}.
    $  
    Esto evita la necesidad de aplicar técnicas de balanceo como \textit{oversampling} o \textit{undersampling}.

    \item \textbf{Distribución y variabilidad de los atributos:}  
    Atributos como \texttt{meanfreq}, \texttt{median}, \texttt{meanfun} y \texttt{centroid} presentan alta variabilidad, lo que corresponde con mediciones continuas del espectro de voz.  
    En contraste, variables como \texttt{mindom} y \texttt{maxdom} muestran menos valores únicos debido a la naturaleza discretizada de la frecuencia dominante.
\end{itemize}

El conjunto de datos presenta buena calidad, no requiere un tratamiento adicional significativo y es adecuado para técnicas de clasificación como Random Forest y Boosting, pero para evitar problemas se realizara una normalizacion en los datos.

\section*{Objetivo de la práctica}
\textbf{Objetivo:} Crear un modelo de clasificación para identificar las características distintivas en los tipos de voz de mujeres y de hombres. Aplique métodos de Random Forest (Bagging) y Boosting.

Agregue la descripción de las variables dependientes e independientes. Explicando el nombre del algoritmo y sus características.

\section*{Tratamiento de datos}
Describa cada una de las variables del conjunto de datos, analice el problema que presentan y aplique la técnica de tratamiento de datos según corresponda. Por ejemplo: tipo, valores faltantes, tratamiento necesario a realizar. Describa el proceso desarrollado.

\section*{Creación del o los modelos de clasificación}
Genere dos modelos de clasificación: Random Forest (Bagging) y Boosting.

\section*{Creación de métricas}
Analice la matriz de confusión para identificar y explicar los resultados encontrados.

Desarrolle las siguientes métricas y su significado:
\begin{itemize}
\item Matriz de confusión
\item Sensibilidad
\item Precisión
\item Tasa de error
\item Exactitud
\item Especificidad
\item Explicación de VP, FP, VN, FN
\item Curva ROC: significado e interpretación
\item Importancia de atributos dentro del modelo
\end{itemize}

\section*{Análisis de datos}
Describa el significado de las métricas generadas a partir del modelo de clasificación.

Identifique dos errores de elementos clasificados erróneamente y describa la razón.

\section*{Conclusiones}
Agregue las conclusiones de la investigación desarrollada.

\section*{Referencias bibliográficas}

\end{document}
