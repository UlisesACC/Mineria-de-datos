
\documentclass[12pt]{article}
\usepackage[margin=1in]{geometry}
\usepackage{longtable}
\usepackage{array}
\usepackage{hyperref}
\usepackage{float}

\begin{document}

\begin{center}
{\LARGE INSTITUTO POLITÉCNICO NACIONAL}\\
{\Large ESCUELA SUPERIOR DE CÓMPUTO}\\
{\Large \textbf{PRACTICA DE CLASIFICACIÓN (Bagging, Boosting)}}\\
\end{center}

\vspace{1cm}

\noindent\textbf{Grupo:} \_\_\_\_\_\_\_\_\\
\textbf{No. Equipo:} \_\_\_\_\_\_\_\_\\
\textbf{Nombres:}
Ulises Abdiel Cabello Cardenas

\vspace{1cm}

\section*{Descripción del conjunto de datos}
Reconocimiento de Género por Voz. Identificar una voz como masculina o femenina.

\noindent\textbf{No. Licencia:} CC BY-NC-SA 4.0\\
\textbf{Disponible en:} \url{https://www.kaggle.com/datasets/primaryobjects/voicegender/data}

\section*{Diccionario de datos}
El conjunto de datos cuenta con 22 atributos, los cuales se describen en la siguiente tabla.

\begin{table}[H]
\centering
\begin{tabular}{lll}
\hline
\textbf{Variable} & \textbf{Tipo} & \textbf{Significado} \\
\hline
meanfreq & numérico & Frecuencia media (kHz) \\
sd & numérico & Desviación estándar de la frecuencia \\
median & numérico & Frecuencia mediana (kHz) \\
Q25 & numérico & Primer cuantil (kHz) \\
Q75 & numérico & Tercer cuantil (kHz) \\
IQR & numérico & Rango intercuartílico (kHz) \\
skew & numérico & Asimetría del espectro \\
kurt & numérico & Curtosis del espectro \\
sp.ent & numérico & Entropía espectral \\
sfm & numérico & Planitud espectral \\
mode & numérico & Frecuencia de la moda \\
centroid & numérico & Centroide de frecuencia \\
peakf & numérico & Frecuencia pico (mayor energía) \\
meanfun & numérico & Promedio de frecuencia fundamental \\
minfun & numérico & Frecuencia fundamental mínima \\
maxfun & numérico & Frecuencia fundamental máxima \\
meandom & numérico & Promedio de frecuencia dominante \\
mindom & numérico & Frecuencia dominante mínima \\
maxdom & numérico & Frecuencia dominante máxima \\
dfrange & numérico & Rango de frecuencia dominante \\
modindx & numérico & Índice de modulación \\
label & categórico & Género: masculino o femenino \\
\hline
\end{tabular}
\end{table}


\section*{Consideraciones encontradas en el conjunto de datos}
Describir las consideraciones que encuentre en el conjunto de datos.

\section*{Objetivo de la práctica}
\textbf{Objetivo:} Crear un modelo de clasificación para identificar las características distintivas en los tipos de voz de mujeres y de hombres. Aplique métodos de Random Forest (Bagging) y Boosting.

Agregue la descripción de las variables dependientes e independientes. Explicando el nombre del algoritmo y sus características.

\section*{Tratamiento de datos}
Describa cada una de las variables del conjunto de datos, analice el problema que presentan y aplique la técnica de tratamiento de datos según corresponda. Por ejemplo: tipo, valores faltantes, tratamiento necesario a realizar. Describa el proceso desarrollado.

\section*{Creación del o los modelos de clasificación}
Genere dos modelos de clasificación: Random Forest (Bagging) y Boosting.

\section*{Creación de métricas}
Analice la matriz de confusión para identificar y explicar los resultados encontrados.

Desarrolle las siguientes métricas y su significado:
\begin{itemize}
\item Matriz de confusión
\item Sensibilidad
\item Precisión
\item Tasa de error
\item Exactitud
\item Especificidad
\item Explicación de VP, FP, VN, FN
\item Curva ROC: significado e interpretación
\item Importancia de atributos dentro del modelo
\end{itemize}

\section*{Análisis de datos}
Describa el significado de las métricas generadas a partir del modelo de clasificación.

Identifique dos errores de elementos clasificados erróneamente y describa la razón.

\section*{Conclusiones}
Agregue las conclusiones de la investigación desarrollada.

\section*{Referencias bibliográficas}

\end{document}
